% !TEX TS-program = pdflatex
% !TEX encoding = UTF-8 Unicode

% This is a simple template for a LaTeX document using the "article" class.
% See "book", "report", "letter" for other types of document.

\documentclass[18pt]{article} % use larger type; default would be 10pt

\usepackage[utf8]{inputenc} % set input encoding (not needed with XeLaTeX)

%%% Examples of Article customizations
% These packages are optional, depending whether you want the features they provide.
% See the LaTeX Companion or other references for full information.

%%% PAGE DIMENSIONS
\usepackage{geometry} % to change the page dimensions
\geometry{letterpaper} % or letterpaper (US) or a5paper or....
% \geometry{margins=2in} % for example, change the margins to 2 inches all round
% \geometry{landscape} % set up the page for landscape
%   read geometry.pdf for detailed page layout information

\usepackage{graphicx} % support the \includegraphics command and options

 \usepackage[parfill]{parskip} % Activate to begin paragraphs with an empty line rather than an indent

%%% PACKAGES
\usepackage{indentfirst}
\usepackage{booktabs} % for much better looking tables
\usepackage{array} % for better arrays (eg matrices) in maths
\usepackage{paralist} % very flexible & customisable lists (eg.enumerate/itemize, etc.)
\usepackage{verbatim} % adds environment for commenting out blocks of text & for better verbatim
\usepackage{subfig} % make it possible to include more than one captioned figure/table in a single float
\usepackage{amsthm}
\usepackage{amsmath}
\usepackage[makeroom]{cancel}
\usepackage{nccmath} 
\usepackage{amssymb}
\usepackage{algorithmic}
\usepackage{natbib}

% These packages are all incorporated in the memoir class to one degree r another...

%%% HEADERS & FOOTERS
\usepackage{fancyhdr} % This should be set AFTER setting up the page geometry
\pagestyle{fancy} % options: empty , plain , fancy
\renewcommand{\headrulewidth}{0pt} % customise the layout...
\lhead{}\chead{}\rhead{}
\lfoot{}\cfoot{\thepage}\rfoot{}

%%% SECTION TITLE APPEARANCE
\usepackage{sectsty}
\allsectionsfont{\sffamily\mdseries\upshape} % (See the fntguide.pdf forfont help)
% (This matches ConTeXt defaults)

%%% ToC (table of contents) APPEARANCE
\usepackage[nottoc,notlof,notlot]{tocbibind} % Put the bibliography in the ToC
\usepackage[titles,subfigure]{tocloft} % Alter the style of the Table of Contents
\renewcommand{\cftsecfont}{\rmfamily\mdseries\upshape}
\renewcommand{\cftsecpagefont}{\rmfamily\mdseries\upshape} % No bold!

%%% END Article customizations

%%% The "real" document content comes below...

\title{BLAH}
\author{BLAH2}
%\date{} % Activate to display a given date or no date (if empty),
         % otherwise the current date is printed

\newtheorem{theorem}{Theorem}
\newtheorem{lemma}{Lemma}


\begin{document}
\maketitle

Suppose a mass $M$ is connected by a string of length $r$ to a point $C$, and
that $M$ is orbiting $C$ with a constant tangential velocity $v$ (in a circle).
What is the acceleration of the mass?

\textbf{PICTURE HERE}

The famous equation for the acceleration is $a = v^2/r$. It's usually
derived by taking the second derivative of the position vector with respect to
time. I figured out a different way to do it, which involves only basic geometry
and computing the average value of $\sin(x)$ with an integral. 

Consider two snapshots of the rotating system at times $t_0$ and $t_1$.


\textbf{IMAGE FOR t0}

\textbf{IMAGE FOR t1}

At $t_0$, the mass's momentum is completely in the $y$ direction. There is no
$x$ component to the momentum. At $t_1$, half a revolution later, the momentum
is in exactly the opposite direction. 

If the momentum at time $t_0$ is $Mv$ and the momentum at time $t_1$ is $M(-v)$,
then the total change in momentum is $-MV - MV = -2MV$. We can get the
\emph{average} change in momentum, \emph{which is the same as the average force
in the $y$ direction} by dividing by the time.

\begingroup
\Large
\begin{align*}
    F_{y_{avg}} &= \frac{-2Mv}{t_1 - t_0} \\
              &=\frac{-2Mv}{\pi r / v} \\
              &=\frac{-2Mv^2}{\pi r}.
\end{align*}
\endgroup

Okay, that tells us the \emph{average} force in the $y$ direction, but that's
not the force we want. We want the radial force. But look! We know how to write
the force in the $y$ direction as a function of the radial force. If the radial
force is $F_s$ and $\theta$ is the angle the string makes with the $y$ axis,
then:

\begingroup
\Large
\begin{align*}
    F_{y} &= F_s\sin(\theta).
\end{align*}
\endgroup

So the \emph{average} of $F_s\sin(\theta)$ over the half-circle of motion will
be equal to the average of $F_y$, which we already know as $F_{y_{avg}}$!

\begingroup
\Large
\begin{align*}
    F_{y_{avg}} &= \frac{\int_{0}^{\pi}F_s\sin(\theta)\mathrm{d}\theta}{\pi} \\
                &= \frac{F_s\int_{0}^{\pi}\sin(\theta)\mathrm{d}\theta}{\pi} \\
                &= \frac{F_s(-\cos(\pi) - -\cos(0))}{\pi} \\
                &= \frac{F_s(1 + 1)}{\pi} \\
                &= \frac{2F_s}{\pi} \\
    \frac{-2Mv^2}{\pi r} &= \frac{2F_s}{\pi} \\
    \frac{-\cancel{2}Mv^2}{\cancel{\pi} r} &= \frac{\cancel{2}F_s}{\cancel{\pi}} \\
    \frac{-Mv^2}{r} &= F_s. \\
\end{align*}
\endgroup

That's it! The radial force $F_s$ is $-Mv^2/r$. Because $F=ma$, the radial
acceleration is therefore $a = -v^2/r$. It is negative because it is towards the
center of the circle. If we are only interested in the magnitude, then $a
= v^2/r$.

Here's a summary of what we did. The general process might be applicable to
other problems.

\begin{enumerate}
    \item Use a simple method to find the average value of some function $f$
          (in this example, the average force).
    \item Write $f$ in terms of some unknown $x$ (in this example, the radial
          force).
    \item Equate the two \emph{averages} to solve for $x$.
\end{enumerate}

\end{document}


